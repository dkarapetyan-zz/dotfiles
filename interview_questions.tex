\documentclass[12pt]{article}
\usepackage[margin=0.5in]{geometry}
\usepackage[latin1]{inputenc}
\usepackage{amsfonts}
\usepackage{amssymb}
\usepackage{enumerate}
\usepackage{hyperref}
\usepackage{framed}
\author{David Karapetyan}
\begin{document}
*Unless otherwise specified, all code to be written in Python, leveraging the
scipy stack (pandas, numpy, statsmodels, scikit-learn, etc.)
\section*{Munging}
\begin{enumerate}[1)]
	\item
		Write code that loads the two weather datasets in the attached h5 file.
	\item
		Munge the weather history and weather forecast dataframes such that
		they contain columns that are either floats or strings. 
	\item Resample the data such that the timestamps occur at 15 minute intervals,
		on the hour (so, for example, 10:00, 10:15, 10:30, 10:45, and so on). 
	\item 
		There is other
		cleaning that needs to be done, but we are deliberately
		being vague. Assume the data will be used to forecast future weather
		patterns. Munge the data further such that it has a higher probability of 
		giving decent results (at the very least, not crashing) a statistical forecasting model on the scipy stack, 
		with comments to your code. There is no perfect answer here--we just want 
		to see how you think.
\end{enumerate}

\section*{Analytics}
\begin{enumerate}[1)]
	\item 
		Write code that computes the weekly means of temperature and humidity. 
		Compute the correlation matrix of weekly means for temperature and humidity
		in the year 2014. 
	\item What does the diagonal of the correlation matrix represent?
		(please write the answer in comments in your code). 
\end{enumerate}
\section*{Forecasting}
\begin{enumerate}[1)]
	\item Using only weather history data from the start of 2015 on, use any
		model of your choosing (from scikitlearn, statsmodels, etc.) to forecast
		the temperature 24 hours into the future (at 15 minute intervals) from the
		terminating time of the weather history data. 
	\item How 'good' is this prediction
		relative to the attached weather forecast dataset?
\end{enumerate}
\end{document}
