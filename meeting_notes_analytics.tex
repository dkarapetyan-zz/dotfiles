\documentclass[12pt]{article}
\usepackage[margin=0.5in]{geometry}
\usepackage[latin1]{inputenc}
\usepackage{amsmath}
\usepackage{amsfonts}
\usepackage{amssymb}
\usepackage{graphicx}
\usepackage{enumerate}
\usepackage{hyperref}
\author{David Karapetyan}
\begin{document}
\section*{DiBoss and TPO Team Decisions}
\begin{itemize}
	\item 2000 original timestamps, which were culled down to 400 of what
		Ashish thought were the most useful.
	\item Then, the building operators/engineers decided which points among
		those were the most import
	\item The BMS has many outages, which results in a string of 0's
		in the time-series
	\item Data integrity issues:
		\begin{enumerate}[(a)]
			\item system shutdowns
			\item data spikes resulting in output $>> 1$
			\item repeating (sticky) values
		\end{enumerate}
	\item bad data is filtered by TPO, and filled via linear interpolation $<2$
		hours on the fly
	\item when TPO is not getting on-the-fly data, it sends the same data
		over and over again to the DiBoss front-end
	\item Selex contract for DiBoss ends June 30.
\end{itemize}
\section*{Columbia/Engineers/John/Neal Meeting 10/2/2015}
\begin{itemize}
	\item Rampdown happens itself, via BMS
	\item Engineers interested primarily in start-up time forecasting and
		occupancy forecasting
	\item Rampdown occurs at 4:00pm by default at the latest--the system will
		pick up on the time people are leaving, and ramp down automatically
	\item Occupancy forecasts are, hence, more for the engineers and manual
		operations
	\item The BMS for 560 Lexington has sensors that measure pressure on air
		ducts (VFD), which signal for fans to slow down (hardware, instantaneous
		calibration--this will beat any statistical forecast, which has, no matter
		how refined, some noise. Over time, noise will cancel itself out, but for
		short times, definitely not. As a result, building costs will be incurred
		in the beginning phases of an installation of the new DiBoss. Hence, VFD
		looks like a good idea--can avoid this initial noise cost by investing in
		cheap hardware)
	\item BMS regulates interior space temperature--so, as long as start time
		is good, we're good (Gene Boniberger). Space temperature by sector--so we
		backwards iterate from sector with highest temperature at first inflection
		point, in order to determine start-up time for whole building. Later, can
		refine so that we have start-up times for each sector, by backwards
		iterating from their individual time-series
	\item However, fan adjustments are not automated in some buildings--some
		may not have VFDs
	\item However, VFDs can always be installed--the real issue is whether
		the VFDs are linked via BMS to the occupancy (Gene)
	\item In the winter months, 6am to 11am is monitored for steam
		emissions--anything above a certain threshold, and Rudin gets an excessive
		charge. Hence, an additional cost function must be provided to the TPO
		model in the winter months. However, this is just an edge case, and can be
		dealt with once we already have a beta of the product
	\item For fan speed, we can assume 100\% instantaneous speed for the fans
		once they are started up. Some buildings do have fans that can be operated
		at a fraction of full capacity, but Gene is looking to just blast all fans
		in a building once start-up is initiated. He'd rather just calibrate
		the actual start-up time, which makes much more sense to me--one start up
		time vs. multiple fans. This is a less-prone-to-error approach
	\item Gene will provide a temperature chart for start-up, to help forecast
		start-up time. This is KEY. More precisely, we will be able to answer the
		following question: how many minutes did it take to get to an inflection
		point in interior space temperature for a given ramp-up time, with outside
		air temperature equal to X?
	\item Startup=startup of fans
	\item People who worked on TPO2: 6 full-time equivalent people (60+ hours
		a week)
	\item Included Ashish, Albert, Roger, and Leon (all full-time)
	\item Total number of employees, full-time and not, was 12.
\end{itemize}
%\section*{Columbia Server Information}
%\begin{itemize}
	%\item{VPN} ID: remote Pass: easyaccess
	%\item{Remote Desktop} ID: ColumbiaAdmin Pass: ColumbiaPassword00
	%\item{Cisco IPSEC VPN} ID: as.aboulanger.com
		%ServerName:	rudin.aboulanger.com
	%\item{SVN} ID: AlbertBoulanger Pass: sk8sk8 Server:
		%\url{https://power.ldeo.columbia.edu/svn/Proj/SmartGrid}
%\end{itemize}
\section*{TPO3}
\begin{itemize}
	\item Hidden Markov Chain implemented by Hooshmand (Biogenetics dept.)
	\item Random Forest Regression implemented by Hooshmand
	\item Ashish's goto expert for statstical questions was Lauren Hanna
\end{itemize}
%\section*{Meeting With Lawyers 06/29/2015}
%\begin{itemize}
	%\item What is exclusive ownership of TPO worth?
	%\item What can we use in DiBoss that we do not have an inventorship claim
		%to?
	%\item Ideal situation is to just have a clean story--may alone make
		%purchasing TPO worthwhile
	%\item Columbia is getting inquiries from 3rd parties
	%\item Decision by lawyers at meeting is to pursue deal to gain access to
		%software and have own the name ``DiBoss''.
	%\item Deal with Columbia would be based on royalties
%\end{itemize}
\section*{Elevators}
\begin{itemize}
	\item Elevators--don't have Selex involvement
	\item Each car has 150 data points
	\item When it gets to the threshold of number of warnings, the car goes
		into maintenance mode
	\item Company gets notified only when the elevator turns off (not good,
		company should be notified, along with engineers, when warning threshold is
		approached or attained)
	\item Gene wants a tab on the front-end displaying elevator data with
		\emph{all} warnings
\end{itemize}
\section*{John's Questions After Return from Italy}
\begin{itemize}
	\item Have you seen the DiBoss source code?
	\item Are you comfortable with the source code?
	\item Are you confident in the support Selex can provide
		for the system, even if it is minimal?
	\item Are you confident in the transfer process of the DiBoss product?
		Of the source code?
\end{itemize}
\section*{Things to Do Differently (Gene)}
\begin{itemize}
	\item Plot scaling is often terrible (due to outlier values). Solution on
		analytics end: filter out outliers before plotting, and offer user option
		to view the outliers in a separate plot
	\item Currently, there is a mislabeling of temperatures, sensors, and
		titles for plots. 6 measurements for the chiller, when only 4 are
		needed by the engineers.
	\item Predict startup-time accurately. \emph{Most Important}
	\item For electric, give a benchmark of best usage, vs. today
	\item Include line with peak usage of electricity, since there is a penalty
		Rudin is assessed by Con-Edison for electricity usage above a certain
		threshold
	\item Lease contract states that system ramp-up must be by 8:00 am (for
		some buildings, it is 7:00). For example--345 Park: 7, 40 East: 7, 80 Pine:
		24/7 on, 1 Battery Park: Monday-Friday, 8:00am-1:00am
\end{itemize}
\section*{TPO3}
\begin{itemize}
	\item HVAC optimization problem. TPO3 is an \emph{optimizer}
	\item TPO2 provided forecasts for steam, electricity, etc and recommended
		a startup time and ramp down time based on a posteriori observations (for
		example, operator behavior in the past)
	\item TPO3 changed things up. No longer depends on past operator behavior,
		like TPO2. Based on optimization instead. Instead, it has its own
		forecaster, the heart of which is an objective function with arguments
		\emph{temperature violations} and \emph{energy costs}
	\item TPO3 is currently running at 345 Park
	\item Training data comes from past operator usage, but forecasts don't
		care about operator actions in present (uh, why should they--SVM and ARMA
		type models operate in a similar fashion)
	\item Trying to predict next state (can use Viterbi, but this is just ARMA
		type with a bit more complexity, perhaps unneeded complexity)
	\item State vectors include VPT frequencies
	\item Hooshmand currently looking into SAX for anomaly detection via
		discretization and clustering. (Cousin to Haar wavelets)
	\item Wanted three months of funding to look at other approaches for
		anomaly detection
\end{itemize}
\section*{Scrum 10/05}
\begin{itemize}
	\item Unless turnstiles are at the bank level, won't know which sector of
		building people are going to based on turnstiles.
	\item 32, 80 Pine are some of
		those where turnstiles DO tell us which sector of building people are
		headed to, which will be critical for interior space temperature
		predictions via occupancy)
	\item Some turnstiles only have ``In'', not an ``Out''--they're just gates
		with an ``open'' and ``closed'' switched
	\item Have access to Mongo University (on Rudin's dime)
\end{itemize}
\section*{Engineering Discussion with Coury and Steve 10/13/2015}
\begin{itemize}
	\item Sunload is an important covariate. For example, 55F with rain feels
		very different than 55 with sunlight, and affects building occupancy.
	\item Startups currently managed by dedicated supervisor on the midnight to
		8 shift, based on space temperatures in the building.
	\item 11:30am-1:30pm is typical interval for rampdown. BMS not reliable for
		this--it is a self-regulating system via hardware components that capture
		physical and chemical phenomena, but NOT occupancy. 	
	\item Wintertime, building is preheated in order to avoid large costs during
		on-demand period of 6am-11am (Heat quantity remains constant in pipes, i.e.
		pressure is constant. So, if heat is being used, more is needed--refilled by Con-Edison). 
	\item Hence, start-up can be done two ways in this
		case: either, simpler problem of optimizing start-up for before 6am, or more
		delicate problem of optimizing for start-up between 6am-11am.
	\item Pneumatic (air) valve controls heat flow on 33rd (key for winter BMS
		self-regulation). Entirely analogous to the VFD
		for controlling fan speed (critical in the summer). 
	\item In short, heat usage regulated automatically via set thermometers
		linked to pneumatic valve. 
\end{itemize}
\section*{Weekly Discussion with Engineers 10/27}
\begin{itemize}
	\item Getting Matt from another building to work part time with team
		identifying pointnames. Will be his top priority (Gene B.), instead of managing
		building.
\end{itemize}
\section*{Radiator Labs}
\begin{itemize}
	\item all custom enclosures per unit
	\item all are insulating--thermostat controlled fan. Condensation rate of
		steam manipulated
	\item when should boiler come on/off
	\item communication protocol is Zigby
	\item steam trap monitoring is part of system
	\item data sent from gateway to Atherios, and a cloud database system (run
		through mongodb)
	\item could easily make an api for Rudin to access specific data points
	\item right now provide temperature and humidity. Looking into leak
		detection
	\item in two columbia buildings (Watt Hall dorms and 47 Claremont), La Fontaine Broadway theatres.
		Upcoming is City Hall in Newark, and Butler Hall
	\item ir thermopile sensor doing averaging across room
	\item can do whatever Nest can do
	\item Occupancy is the next step
\end{itemize}

\section*{Weekly Scrum Meeting 11/3/2015}
\begin{itemize}
	

\item Backend looking into getting elevator information, but need api

\end{itemize}
\end{document}
