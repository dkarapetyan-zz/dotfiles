\documentclass[12pt]{article}
\usepackage[margin=0.5in]{geometry}
\usepackage[latin1]{inputenc}
\usepackage{amsfonts}
\usepackage{amssymb}
\usepackage{enumerate}
\usepackage{hyperref}
\usepackage{framed}

\begin{document}

\author{David Karapetyan \\ Lead Data Scientist \\ PrescriptiveData}
\date{}
\title{How Machine Learning is Changing Decades of Building Management}
\maketitle

\subsection*{Introduction}
Efficient energy management is a growing concern as urban centers
become overcrowded, and the cost of utilities rises. Commercial office spaces
have historically relied upon teams of engineers to calibrate building energy
usage based on outside weather conditions, building occupancy, and general
intuition. While this approach has served well over the years, engineers are
looking to more modern methods of energy management to optimize energy usage,
and scale to the demands of a growing urban workforce.

PrescriptiveData and others have a number of research results which improve energy usage benchmarks via the introduction of a proprietary machine learning analytics engine that targets the
following: 

\begin{itemize}
	\item{\textbf{Occupancy Forecasts}}. Commercial building energy usage is, on a given
		day, governed to a large extent by its occupancy. As a simple example, 
		a building's energy usage could be scaled down near noon depending on 
		the number of people expected to leave the building for lunch hours. The
		engine, given reliable data, provides accurate forecasts of occupancy
		evolution for one day into the future from the present. 
	
	\item{\textbf{Utility Usage Forecasts and Benchmarks}}. Daily utility usage (steam,
		electricity, water) for commercial
		office spaces is computed, with predictions improving over the course of the
		year as more and more reliable
		data is processed from proprietary sensors installed throughout client
		buildings. More precisely, the more clean data the engine has available to 'learn'
		from, the better the predictions.
		A benchmark of best past management of individual utilities is also computed
		and provided to the building engineers as a means of evaluating 
		engineer and hardware job performance, and cross-validating the engine
		performance against itself. 

	\item{\textbf{BMS Start Up Times}}. Commercial Buildings have an internal
		regulating building management system (BMS) provided by an integrator which is started up in the early morning, and which has
		a system of rules for regulating building outside air intake, steam
		propagation, and much more.  Optimal start up time is computed, based on
		past engineer choices for startup time. 
	
	\item{\textbf{Operation in the Cloud}} Cloud computing has emerged as a
		relatively inexpensive, scalable and more flexible
		alternative to past local hardware solutions. 
		A cloud implementation allows a suite of applications to be deployed
		and monitored with ease. Furthermore, large computations can executed on not
		one but a cluster of several processors, 
		the number of which can be increased or decreased
		dynamically, depending on client needs and energy demand. 

\end{itemize}
\subsection*{Why?}
Building engineers have accumulated a wealth of experience over years of
operating a wide variety of commercial buildings. However, human operation of
commercial spaces has a number of downsides:
\begin{itemize}
	\item{\textbf{Fatigue}} Unlike machines, humans are prone to fatigue,
		resulting in human error and potentially large losses (especially in the
		wintertime, when steam prices are at an annual peak).
	\item{\textbf{Human Error}} Even without fatigue, human decision making is
		prone to error. PrescriptiveData's benchmarks and data analysis indicate
		sometimes severe errors throughout a given year.
	\item{\textbf{Qualitative Decision Making}} Building engineers often operate
		on intuition, or 'feel', without rigorous quantitative metrics for choices
		regarding building management. In the long run, this results in
		inefficient management of the building, and unnecessary incurred costs.
	\item{\textbf{Labor Costs}} Roughly the aggregate of two building engineers' annual
		salaries can be used to purchase the equipment necessary to execute the
		analytics engine. While there is a monetary cost in
		software development time and labor, over the long run it is less than that
		of maintaining a team of engineers to manage a building. 

	

\end{itemize}
\subsection*{Future Efforts}
\subsubsection*{A Unified Platform For Predictions and Building
Systems}
Currently, PrescriptiveData's efforts are centered around providing reliable
predictions and benchmarks for 345 Park Avenue, a large commercial office space
in New York City. Future efforts will focus on providing the
analytics engine for all the commercial buildings in Rudin Management's
portfolio, and a sleek user interface integrating all building system states,
benchmarks, and predictions. In order to properly scale with the rapidly increasing influx of data, both the engine implementation and user interface will evolve along with
the global software ecosystem in order to continually provide a responsive
interface and fast predictions and metrics for clients.

PrescriptiveData's vision for an intelligently managed commercial office space is one
that incorporates the latest developments in machine learning and mathematics
with the most robust and scalable software engineering developments currently
available in the open source ecosystem. The power of this approach will be made
evident via:
\begin{enumerate}[1)]
	\item Increased energy savings for commercial buildings by optimizing utility
		usage.
	\item Decreased labor costs. Only a small number of engineers will be
		required, mostly acting as a failsafe.
	\item Reduction of the commercial building's carbon footprint by eliminating
		energy waste
	\item Easier building monitoring for owners and managers. This monitoring will
		be through an intuitive graphical user interface that exposes
		building benchmarks, current state, and machine learning predictions for
		future state.
\end{enumerate}
With increasing adoption of the
software platform by other clients, the engine will have enough data coverage to
not only make forecasts about energy usage in commercial office spaces, but a
wide variety of macroeconomic phenomena within their neighborhoods and, over
time, the city itself. 

\subsection*{Author Bio}
\textit{David Karapetyan is the lead data scientist at PrescriptiveData, a company
making commercial buildings smarter by providing an integrated analytics and
machine learning engine. He received his doctorate in mathematics from the
University of Notre Dame, and has extensive experience in mathematical modeling
and simulation.}
\end{document}
