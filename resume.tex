\documentclass{resume}
\usepackage{pdfpages}
\newsectionwidth{1cm}
\setlength{\textheight}{11in} 
\usepackage[top=0.75in, bottom=0.75in, left=0.75in, right=1.25in]{geometry}  %page layout
\usepackage{hyperref}
% \hypersetup{colorlinks=true, linkcolor=black, citecolor=black, urlcolor=black}
\begin{document} 
\name{
    \Large DAVID KARAPETYAN \\
}     
\address{ 
\href{mailto:david.karapetyan@gmail.com}{\nolinkurl{david.karapetyan@gmail.com}}
\\
\url{http://davidkarapetyan.com}}
\begin{resume}
\section{EXPERIENCE}
	\vspace{-0.05mm}

\begin{tabbing}
\hspace{2.3in}\= \hspace{2.6in}\= \kill
{\bf Head Data Scientist} \> Prescriptive Data   
\>\textbf{June 2015--Present } \\
    \>New York City, NY 
\end{tabbing}
$\bullet$
Leading team that is designing and implementing a \textbf{machine learning
forecasting \\ and analytics engine} for energy usage in commercial and non-commercial
buildings for Rudin Management, a leading real estate developer in NYC.  \\
$\bullet$ Using weather and other proprietary covariates, engine forecasts: optimal building start-up-time, ramp-down-time, 
daily building steam usage, daily electricity usage, occupancy, and others. \\
$\bullet$ 
Greatly improved upon building predictions made by a team of Columbia University
Ph.D and masters degree statisticians (a three year project that was ultimately
rejected due to poor results) based upon a number of different metrics,
including: classification cross-validation accuracy, random forest out-of-bag
scores, mean and variance of residuals, max/min residuals, and others. \\
$\bullet$
Using weather and other proprietary covariates, engine forecasts: optimal building start-up-time, ramp-
down-time, daily building steam usage, daily electricity usage, and occupancy.
\\
$\bullet$
Implementation uses \textbf{parallel programming in Python}, 
with modules including \textbf{scikit-learn, pandas, matplotlib, statsmodels}
and \textbf{numpy}.
Data is stored in and read from \textbf{SQL}, \textbf{MongoDB}, and \textbf{HDF5} dataframes.\\
$\bullet$
Models used for the forecasting include \textbf{Random Forests}, \textbf{Gradient
Boosted Trees}, \textbf{ARIMA(X)}, \\ \textbf{SARIMA(X)}, and \textbf{SVM}. 
\begin{tabbing}
\hspace{2.3in}\= \hspace{2.6in}\= \kill
{\bf Quantitative Analyst} \> Ernst \& Young   
\>\textbf{June 2014--June 2015 } \\
    \>New York City, NY 
\end{tabbing}
$\bullet$
Developed Class Model forecasting module in \textbf{R}. Used
\textbf{ARIMA regression \\ on macroeconomic scenarios} (base, adverse, or severely adverse) and position data to forecast and plot any input bank’s PPNR,
Provision, Capital and other variables with respect to time.  \\
$\bullet$ Provide valuation and advanced financial modeling expertise to
institutional clients in regards to complex securities including equity and
foreign exchange options, rates swaptions, and related embedded derivative
instruments.  \\
$\bullet$ 
Analyzed \textbf{Monte Carlo and Finite Difference models} to determine fair value of
client instruments for accounting purposes.  \\
$\bullet$ Designed and performed \textbf{stress-tests} for investment bank
client’s pricing models for \textbf{CCAR} purposes. Evaluated the impact on
\textbf{PV and option Greeks} of client’s portfolio of equity and foreign exchange exotic instruments under severely adverse market scenarios. 
\\
$\bullet$
Provide data analysis of trade desk definitions and descriptions, and report
anomalies to client. Trades included \textbf{forex USD and G10 pairs, G10 and
	emerging market pairs, trades with long and short expiry, Asian options,
barriers}, and a variety of others.
\begin{tabbing}
\hspace{2.3in}\= \hspace{2.6in}\= \kill
{\bf Visiting Assistant Professor} \>University of Rochester     
\>\textbf{July 2012--July 2014} \\
    \>Rochester, NY 
\end{tabbing}
$\bullet$ Researcher of partial differential equations, in particular nonlinear evolution equations.  
\\
$\bullet$ Publications list with doctoral thesis at \url{http://davidkarapetyan.com/pdfs/publications.pdf}
\\
$\bullet$ Developed \textbf{numerical simulations in C++}  to gain
intuition about whether certain  equations are well-posed
or ill-posed for rough initial data. \\
$\bullet$ Code included Doolittle factorizations and \textbf{finite
difference schemes with spline interpolation}. \\
$\bullet$ Taught courses on Numerical Analysis, Calculus,
Topology, Applied Mathematics, Linear Algebra,
Differential Equations, and Financial Mathematics.
\begin{tabbing}
\hspace{2.3in}\= \hspace{2.6in}\= \kill
{\bf Research Associate } \> Institute for
Defense Analyses \>  
\textbf{Aug 2004--Aug 2005}\\
    \>Alexandria, VA
\end{tabbing}
$\bullet$ Conducted research for Pentagon sponsored and privately
sponsored projects. 
\\ 
$\bullet$ Applied \textbf{$k$-Nearest Neighbor regression} to analyze existing data on high
occupancy toll lanes in the Los
Angeles, San Diego, and
Chicago metro areas.
\section{EDUCATION} 
\vspace{-0.05mm}
\begin{tabbing}
\hspace{2.3in}\= \hspace{2.6in}\= \kill
\textbf{University of Notre Dame} \\
$\bullet$ \textbf{Ph.D}, Mathematics \>\>\textbf{Aug 2007--May 2012}
\\ 
Thesis: \textit{On the well-posedness of the hyperelastic rod equation} \\
$\bullet$ Awarded the \textbf{Schmitt Presidential Fellowship}. Full scholarship. \\
\hspace{2.3in}\= \hspace{2.6in}\= \kill
\\
\textbf{University of California, Berkeley}
\\
$\bullet$ \textbf{B.S.}, Mathematics  and  \textbf{B.A.}, English Literature.
\>\>\textbf{Aug 2000--May 2004}
\\
$\bullet$ Awarded the \textbf{Regents Scholarship}. Full scholarship.
\\
$\bullet$ Nominee for ND Shaheen Graduate School Award for top student.
\end{tabbing}
\section{TECHNICAL SKILLS}
\vspace{-0.05mm}
\begin{tabbing}
$\bullet$ Languages: Python (full SciPy stack, Flask), Scala,  R, C/C++,
HTML5, Bash, \LaTeX{}.
\\
$\bullet$ Operating Systems: Unix (Debian/Ubuntu, FreeBSD, OS X) , Windows (XP, Vista, 7) 
\end{tabbing}

\section{HONORS, AWARDS, AND EXTRACURRICULAR ACTIVITIES} 
\vspace{-0.05mm}
\begin{tabbing}
\hspace{2.3in}\= \hspace{2.6in}\= \kill
$\bullet$ Chess Expert \url{http://www.chessdryad.com/articles/mi/article_165.htm}
\end{tabbing}
\end{resume}
%\includepdf{cover-letter-editor.pdf}
\end{document}
